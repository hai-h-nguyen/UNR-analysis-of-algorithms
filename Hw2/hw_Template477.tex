\documentclass[[11pts]{report}

\usepackage{qtree}
\usepackage{listings}
\usepackage{amsmath,mathtools}
\usepackage[ruled,longend]{algorithm2e}
\usepackage{tikz}

\title{CS 677 Homework \\ Assignment 02}
\date{Semtember 11, 2018}
\author{Hai Nguyen}

\setlength{\topmargin}{-1cm}
\setlength{\oddsidemargin}{0in}
\setlength{\textwidth}{6.5in}
\setlength{\textheight}{8.3in}

\DeclareMathOperator{\Div}{div}
\newcommand{\vect}[1]{\mathbf{#1}}




%%Currently default settings for indentation and symbols.
%%Try these by uncommenting this block!!!
%%Redefine the first level symbols
%\renewcommand{\theenumi}{\fnsymbol{enumi}-}
%\renewcommand{\labelenumi}{\theenumi}
%
%%Redefine the second level symbols
%\renewcommand{\theenumii}{\alph{enumii})}
%\renewcommand{\labelenumii}{\theenumii}
%
%%Redefine the third level symbols
%\renewcommand{\theenumiii}{\roman{enumiii}.}
%\renewcommand{\labelenumiii}{\theenumiii}
%
%%Options for redefining levels


%\arabic
%\alph 
%\Alph
%\roman
%\Roman
%\fnsymbol
%This ^^^ is all you need to change!!

\begin{document}

\maketitle

\begin{enumerate}

\item Consider the following recursive algorithm.

\textbf{ALGORITHM} \textit{Min1}(A[0..n-1])
\\//Input: An array A[0..n-1] of integer number
\\\textbf{if} n = 1 return A[0]
\\\textbf{else} \textit{temp} $\gets$ \textit{Min1}(A[0..n-2])
\par \quad \quad \textbf{if} \textit{temp} $\leq$ A[n-1] \textbf{return} \textit{temp}
\par \quad  \quad \textbf{else} \textbf{return} A[n-1]

\begin{enumerate}
\item What does this algorithm compute? 
\begin{center}
\textbf{Minimum of an array of integer numbers.}
\end{center} 

\item Set up a recurrence relation for the algorithm's basic operation count and solve it
\par \textbf{ALGORITHM} \textit{Min1}(A[0..n-1])
\par \textbf{if} n = 1 return A[0]  \quad [Constant time c1]
\par\textbf{else} \textit{temp} $\gets$ \textit{Min1}(A[0..n-2])  \quad  [c2 + T(n-1)]
\par \quad \quad \textbf{if} \textit{temp} $\leq$ A[n-1] \textbf{return} \textit{temp} \quad [Constant time c3]
\par \quad  \quad \textbf{else} \textbf{return} A[n-1] \quad [Constant time c4]
\\ Recurrent relationship:
\begin{align*}
T(1) &= c1 \\
T(n) &= c + T(n-1)
\end{align*}
Solve it:
\begin{align*}
T(n) &= c + T(n-1) \\
     &= T(1) + (n-1)c \\
     &= \Theta(n)
\end{align*}
\end{enumerate}

\item Solve the following recurrences using the method of your choice.

\begin{itemize}

\item $T(n) = 2T(\frac{n}{2}) + n^3$
\\ \textit{Iteration method:}
\\ Assume: $n = 2^k$
\begin{align*}
T(n) &= 2T(\frac{n}{2}) + n^3 = 2(2T(\frac{n}{4}) + (\frac{n}{2})^3) + n^3 = 4T(\frac{n}{4}) + n^3(1 + \frac{1}{4}) \\ &= 4(2T(\frac{n}{8}) + (\frac{n}{4})^3) + n^3(1 + \frac{1}{4}) = 2^3T(\frac{n}{2^3}) + n^3(1 + \frac{1}{4} + \frac{1}{4^2}) \\ &= ... \\ &= 2^iT(\frac{n}{2^i}) + n^3(1 + \frac{1}{4} + \frac{1}{4} +  ... + \frac{1}{4^{i-1}}) \\ &= nT(1) + n^3 \sum_{i=1}^{lgn}{\frac{1}{4^{i-1}}} \\ &= c_1n + c_2n^3  \\ &= \Theta(n^3)
\end{align*}

\item $T(n) = T(\sqrt{n}) + 1$
\\ Rename: $m = lgn \to n = 2^m$
\\ $T(2^m) = T(2^{m/2}) + 1$
\\ Rename: $S(m) = T(2^m)$

\begin{align*}
S(m) &= S(m/2) + 1 \\
	 &= S(m/4) + 2 \\
	 &= S(m/2^i) + i \\
	 &= \hdots \\
	 &= S(1) + lgm \\
	 &= \Theta(lgm)
\end{align*}
\\ $S(m) = S(m/2) + 1 = \Theta(lgm)$
\\ 
\\ $T(n) = T(2^m) = S(m) = \Theta(lgm) = \Theta(lglgn)$

\item $T(n) = 3T(\frac{n}{2}) + nlgn$
\\ \textit{Master's method:}
$a = 3, b = 2, f(n) = nlgn$.
\par Compare $n^{\log_b a} \approx n^{1.58}$ and $nlgn$, for some constant $\epsilon > 0$:

\begin{equation*}
f(n) = nlgn = O(n^{1.58 - \epsilon})
\end{equation*}
$\to$ Case 1:
\begin{equation*}
 T(n) = \Theta(n^{\log_2 3})
\end{equation*}


\end{itemize}

\item

\begin{itemize}

\item Draw the recursion tree for $T(n) = T(n/4) + T(n/2) + n^2$ and provide a tight asymptotic bound on its solution.


\begin{tikzpicture}[level/.style={sibling distance=60mm/#1}]
\node [circle,draw] (z){$n^2$}
  child {node [] (a) {$(\frac{n}{4})^2$}
    child {node [] (b) {$(\frac{n}{4^2})^2$}
      child {node {$\vdots$}
        child {node [] (d) {$T(1)$}}
        child {node [] (e) {$T(2)$}}
      } 
      child {node {$\vdots$}}
    }
    child {node [] (g) {$(\frac{n}{8})^2$}
      child {node {$\vdots$}}
      child {node {$\vdots$}}
    }
  }
  child {node [] (j) {$(\frac{n}{2})^2$}
    child {node [] (k) {$(\frac{n}{8})^2$}
      child {node {$\vdots$}}
      child {node {$\vdots$}}
    }
  child {node [] (l) {$(\frac{n}{4})^2$}
    child {node {$\vdots$}}
    child {node (c){$\vdots$}
      child {node [] (o) {$\vdots$}
      	child {node [] (or) {$\vdots$}}
      	child {node [] (ol) {$\vdots$}}
      	}
      child {node [] (p) {$\vdots$}
      	child {node [] (pr) {$T(1)$}}
      	child {node [] (pl) {$T(2)$}
			child [grow=right] {node (q) {$(\frac{5}{16})^Mn^2$} edge from parent[draw=none]
				child [grow=up] {node (r) {$(\frac{5}{16})^{M-1}n^2$} edge from parent[draw=none]
					child [grow=up] {node (r1) {$\hdots$} edge from parent[draw=none]
						child [grow=up] {node (r) {$(\frac{5}{16})^2n^2$} edge from parent[draw=none]
							child [grow=up] {node (r) {$(\frac{5}{16})^1n^2$} edge from parent[draw=none]
								child [grow=up] {node (r) {$(\frac{5}{16})^0n^2$} edge from parent[draw=none]
									child [grow=up] {node (r) {\textbf{Total Cost}} edge from parent[draw=none]}
								}
							}
						}
					}		
				}
			}      	
      	}      
      }
    }
  }
};

\end{tikzpicture}
\par Unbalanced tree, left branch ends sooner. Tight bound is a new balanced tree with the left branch replaced by the reflection of the old right branch.
\begin{align*}
T(n) \leq n^2 (1 + (\frac{5}{16})^2 + ... + (\frac{5}{16})^M)
\end{align*}
With:
\begin{equation*}
\frac{n}{2^M} = 2
\end{equation*}
Because $\frac{5}{16} < 1$, we have:
\begin{align*}
T(n) &\leq n^2 (1 + (\frac{5}{16})^2 + ... + (\frac{5}{16})^M) \\
     &\leq n^2\frac{1}{1-\frac{5}{16}} \\
     &= \mathbf{O(n^2)}
\end{align*}


\item Use the iteration method to solve the following recurrence:
\\ $T(n) = 4T(n/2) + n$
\\ Assume: $n = 2^k$
\begin{align*}
T(n) &= 4T(n/2) + n \\
&= 4(4T(n/2^2) + n/2) + n \\
&= 4^2T(\frac{n}{2^2}) + 4\frac{n}{2} + n \\
&= 4^3T(\frac{n}{2^3}) + 4^2\frac{n}{2^2} +  4\frac{n}{2} + n \\
&= \hdots \\
&= 4^iT(\frac{n}{2^i}) + n(1 + 2^1 + ... + 2^{i-1}) \\
&= \hdots \\
&= 4^{lgn}T(1) + n\sum_{i=0}^{lgn-1}2^i \\
&= n^2T(1) + n(n - 1) \\
&= \mathbf{\Theta(n^2)}
\end{align*}


\end{itemize}

\item Considering the following recursive algorithm:
\\\textbf{ALGORITHM Q(n)}
\par //Input: A positive integer n
\\\textbf{if} $n=1$
\par \quad return 1
\par \textbf{else}
\par \quad return \textbf{Q(n-1) + 2n - 1}

\begin{itemize}
\item Set up a recurrence relation for this function's value and solve it to determine what this algorithm computes:
Recurrence relation:
\begin{align*}
Q(n) &= Q(n-1) + 2n-1 \: \text{With} \: \forall n > 1 \\
Q(1) &= 1
\end{align*}
Solve it:
\begin{align*}
Q(n) &= Q(n-1) + 2n-1 \: \text{With} \: \forall n > 1 \\
     &= 1 + 3 + 5 + ... + 2n - 1 \\
     &= \mathbf{n^2 }
\end{align*}

\item Set up a recurrence relation for the number of multiplications made by this algorithm and solve it.

\par \textbf{ALGORITHM Q(n)}
\par //Input: A positive integer n
\par \textbf{if} $n=1$
\par \quad return 1
\par \textbf{else}
\par \quad return \textbf{Q(n-1) + 2n - 1}
\par Recurrence relation for the number of multiplications:
\begin{align*}
M(1) &= 0 \\
M(n) &= M(n-1) + c
\end{align*}
\par Solve it:
\begin{align*}
M(n) &= M(n-1) + 1 \\
	 &= M(1) + (n-1)1 \\
	 &= \mathbf{\Theta(n)}
\end{align*}
The total number of multiplications made: $\mathbf{M(n) = n - 1}$


\end{itemize}

\item Consider the following algorithm.

\par \textbf{ALGORITHM} \textit{Mystery}(n)
\par //Input: A non-negative integer n
\par $S \gets 0$
\par \textbf{for} $i \gets 1$ \textbf{to} n \textbf{do}
\par \quad $S \gets S + i*i$
\par \textbf{return} S

\begin{itemize}
\item What does this algorithm compute?
\begin{center}
$\mathbf{\sum_{i=1}^ni^2}$
\end{center}
\item Compute the running time of this algorithm.

$S \gets 0$        [Cost: c1, Times: 1] \\
\textbf{for} $i \gets 1$ \textbf{to} n \textbf{do} [Cost: c2, Times: n + 1]
\par \quad $S \gets S + i*i$ [Cost: c3, Times: n] \\
\textbf{return} S [Cost:c4, Times:1]
\\ \\ Total cost: $c1 + c2 \times (n+1) + c3 \times n + c4 = \mathbf{\Theta(n)} $
\end{itemize}

\end{enumerate}


\end{document}
