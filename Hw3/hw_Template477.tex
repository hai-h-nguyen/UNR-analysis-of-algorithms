\documentclass[11pts]{report}

\usepackage{qtree}
\usepackage{listings}
\usepackage{amsmath,mathtools}
\usepackage[ruled,longend]{algorithm2e}
\usepackage{tikz}
\usepackage{listings}
\usepackage{xcolor}
\lstset{
    frame=tb, % draw a frame at the top and bottom of the code block
    tabsize=4, % tab space width
    showstringspaces=false, % don't mark spaces in strings
    commentstyle=\color{green}, % comment color
    keywordstyle=\color{blue}, % keyword color
    stringstyle=\color{red} % string color
}

\title{CS 677 Homework \\ Assignment 03}
\date{Semtember 18, 2018}
\author{Hai Nguyen}

\setlength{\topmargin}{-1cm}
\setlength{\oddsidemargin}{0in}
\setlength{\textwidth}{6.5in}
\setlength{\textheight}{8.3in}

\DeclareMathOperator{\Div}{div}
\newcommand{\vect}[1]{\mathbf{#1}}




%%Currently default settings for indentation and symbols.
%%Try these by uncommenting this block!!!
%%Redefine the first level symbols
%\renewcommand{\theenumi}{\fnsymbol{enumi}-}
%\renewcommand{\labelenumi}{\theenumi}
%
%%Redefine the second level symbols
%\renewcommand{\theenumii}{\alph{enumii})}
%\renewcommand{\labelenumii}{\theenumii}
%
%%Redefine the third level symbols
%\renewcommand{\theenumiii}{\roman{enumiii}.}
%\renewcommand{\labelenumiii}{\theenumiii}
%
%%Options for redefining levels


%\arabic
%\alph 
%\Alph
%\roman
%\Roman
%\fnsymbol
%This ^^^ is all you need to change!!

\begin{document}

\maketitle

\begin{enumerate}

% Question 1
\item Consider the following recursive algorithm.

\textbf{ALGORITHM} \textit{Enigma}(A[0..n-1])
\\//Input: An array A[0..n-1] of integer number
\par \textbf{for} $i \gets 0$ \textbf{to} n - 2 \textbf{do}
\par \quad \textbf{for} $j \gets i+1$ \textbf{to} n - 1 \textbf{do}
\par \quad \quad \textbf{if} $A[i] == A[j]$
\par \quad \quad \quad \textbf{return false}
\par \textbf{return true}

\begin{enumerate}

\item What does this algorithm do?
\par Check for 2 duplicate elements within an array of integer numbers. If there is at least two duplicate elements, the algorithm returns False, otherwise returns True.

\item Compute the running time of this algorithm.

\textbf{ALGORITHM} \textit{Enigma}(A[0..n-1])
\par \textbf{for} $i \gets 0$ \textbf{to} n - 2 \textbf{do} \textit{[Cost c1, n times]}
\par \quad \textbf{for} $j \gets i+1$ \textbf{to} n - 1 \textbf{do} \textit{[Cost c2, n - i times]}
\par \quad \quad \textbf{if} $A[i] == A[j]$ \textit{[Cost c3, n - i - 1 times, worst case]}
\par \quad \quad \quad \textbf{return false}
\par \textbf{return true}
\par Total cost:
\begin{align*}
T(n) &\leq c_1n + \sum_{i=0}^{n-2}(n-i) + \sum_{i=0}^{n-2}(n-i-1) \\
	 &= c_1n + \sum_{k=2}^{n}k + \sum_{k=1}^{n-1}k \\
	 &= O(n^2)
\end{align*}
\end{enumerate}

\item 

\begin{enumerate}
\item Implement in C/C++ a version of bubble sort that alternates left-to-right and right-to-left passes through the data.

Source code:
\begin{lstlisting}
#include <stdio.h>

int numComp = 0;
 
void swap(int *xp, int *yp)
{
    int temp = *xp;
    *xp = *yp;
    *yp = temp;
}

/* Function to print an array */
void printArray(int arr[], int size)
{
    int i;
    for (i=0; i < size; i++)
        printf("%c ", arr[i]);
}

void bubbleSort(int arr[], int n)
{
   int i, j;
   int leftStartIdx = 1;
   int rightStartIdx = n; 

   while (leftStartIdx < rightStartIdx)
   {
       // Left2Right pass
         for (j = leftStartIdx; j <= rightStartIdx - 1; j++)
         {
            numComp++;

            if (arr[j-1] > arr[j])
            {
               swap(&arr[j], &arr[j-1]);
            }
         }
        

        rightStartIdx  -= 1;


        printf("Left2Right pass: ");
        printArray(arr, n);
        printf("\n");        

        if (leftStartIdx >= rightStartIdx)
        	break;

        // Right2Left pass
         for (j = rightStartIdx - 1; j >= leftStartIdx; j--)
         {
            numComp++;

            if (arr[j - 1] > arr[j])
            {
               swap(&arr[j - 1], &arr[j]);

            }
         }
         
         printf("Right2Left pass: ");
         printArray(arr, n);
         printf("\n");

         if (leftStartIdx >= rightStartIdx)
    		  break;

    	leftStartIdx += 1;
   }
}
 
int main()
{
    int arr[] = {'E', 'A', 'S', 'Y', 'Q', 'U', 'E', 'S', 'T', 'I', 'O', 'N'};
    int n = sizeof(arr)/sizeof(arr[0]);
    bubbleSort(arr, n);
    return 0;
}

\end{lstlisting}
\par 
\par
Outputs:

\begin{lstlisting}
Left2Right pass: A E S Q U E S T I O N Y 
Right2Left pass: A E E S Q U I S T N O Y 
Left2Right pass: A E E Q S I S T N O U Y 
Right2Left pass: A E E I Q S N S T O U Y 
Left2Right pass: A E E I Q N S S O T U Y 
Right2Left pass: A E E I N Q O S S T U Y 
Left2Right pass: A E E I N O Q S S T U Y 
Right2Left pass: A E E I N O Q S S T U Y 
Left2Right pass: A E E I N O Q S S T U Y 
Right2Left pass: A E E I N O Q S S T U Y 
Left2Right pass: A E E I N O Q S S T U Y 
\end{lstlisting}

\item How many comparisons does this modified version of bubble sort make?
\begin{itemize}
\item First Left2Right pass: We have n elements, and it takes us n - 1 comparisons
\item First Right2Left pass: After the first left2right pass, we only have n-1 elements, and it takes us n - 2 comparisons
\item We continue until we only have 2 elements and we only need to make 1 comparions
\par Total comparisons needed:
\begin{equation*}
1 + 2 + 3 + \hdots + n - 1 = \frac{n(n-1)}{2}
\end{equation*}
\end{itemize}
\end{enumerate}

\item Non-recurisve merge sort
\par Code:

\begin{lstlisting}
#include <stdio.h>

void printArray(int arr[], int size, bool c = false)
{
    int i;
    for (i=0; i < size; i++)
    {
        if (c)
            printf("%c ", arr[i]);
        else
            printf("%d ,", arr[i]);
    }
    printf("\n");
}

void merge(int arr[], int l, int m, int r) 
{ 
    int i, j, k; 
    int n1 = m - l + 1; 
    int n2 =  r - m; 
  
    int L[n1], R[n2]; 
  
    for (i = 0; i < n1; i++) 
        L[i] = arr[l + i]; 
    for (j = 0; j < n2; j++) 
        R[j] = arr[m + 1 + j]; 
  
    i = 0;  
    j = 0; 
    k = l; 
    while (i < n1 && j < n2) 
    { 
        if (L[i] <= R[j]) 
        { 
            arr[k] = L[i]; 
            i++; 
        } 
        else
        { 
            arr[k] = R[j]; 
            j++; 
        } 
        k++; 
    } 
  
    while (i < n1) 
    { 
        arr[k] = L[i]; 
        i++; 
        k++; 
    } 
  
    while (j < n2) 
    { 
        arr[k] = R[j]; 
        j++; 
        k++; 
    } 
} 

void nonRecursiveMerge (int arr [], int n)
{
    int m = 1;
    int i = 0;
    int minVal = 0;

    while (m < n)
    {
        i = 0;
        while (i < n - m)
        {
            minVal = (i + 2 * m - 1 < n - 1) ? (i + 2 * m - 1) : (n - 1);
            merge(arr, i, i + m - 1, minVal);
            i += 2*m;
        }

        printArray(arr, n, true);

        m *= 2;
    }
}

int main()
{
    int arr[] = {'A', 'S', 'O', 'R', 'T', 'I', 'N', 'G',
     'E', 'X', 'A', 'M', 'P', 'L', 'E'};
    int n = sizeof(arr)/sizeof(arr[0]);
    nonRecursiveMerge(arr, n);
    return 0;
}
\end{lstlisting}


Output:
\begin{lstlisting}
A S O R I T G N E X A M L P E 
A O R S G I N T A E M X E L P 
A G I N O R S T A E E L M P X 
A A E E G I L M N O P R S T X
\end{lstlisting}


\item Use a loop invariant to prove that the following algorithm computes $a^n$:

\par Exp(a, n)
\par \{
\par \qquad i $\gets$ 1
\par \qquad pow $\gets$ 1
\par \qquad while (i $\leq$ n)
\par \qquad \{
\par \qquad \qquad pow $\gets$ pow*a
\par \qquad \qquad i $\gets$ i + 1
\par \qquad \}
\par \qquad return pow
\par \}

\par Use the following loop invariant:
\begin{equation*}
\text{pow}_i = \text{a}^i
\end{equation*}
Prove the loop invariant:

\begin{itemize}
\item Initialization
\par i = 0: $\text{pow}_0$ = $a^0$ = 1

\item Maintenance:
Assume that at the start of the i-th iteration $\text{pow}_i = \text{a}^i$
\par Then, at the start of the (i+1)-th iteration we will have:
$\text{pow}_{i+1} = \text{pow}_i\times a = \text{a}^{i+1}$
\item Termination: The loop terminate when i = n. Thus after the loop execution we have:
\begin{equation*}
\text{pow}_n = \text{a}^n
\end{equation*}

\end{itemize}

\item Consider another algorithm for solving the same problem as the one in Homework 2 (problem 1), which recursively divides an array into two halves (call Min2 (A[0...n-1])):

\begin{enumerate}
\item Set up a recurrence relation for the algorithm's basic operation count and solve it
\par \textbf{ALGORITHM} \textit{Min2}(A[\textit{left}..\textit{right}])
\par \quad \textbf{if} \textit{left = right}  \textbf{return} A\textit{[left]} \quad \textit{[Cost c1]}
\par \quad \textbf{else} \textit{temp1} $\gets$ Min2(A[\textit{left}..(\textit{left} + \textit{right})/2) \textit{[Cost $Theta(n/2)$]}
\par \quad \quad \textit{temp2} $\gets$ Min2(A[(\textit{left} + \textit{right})/2+1..\textit{right}) \textit{[Cost $Theta(n/2)$]}
\par \quad \quad \textbf{if} \textit{temp1} $\leq$ \textit{temp2} \textbf{return} \textit{temp1} \textit{[Cost c2]}
\par \quad \quad \textbf{else return} \textit{temp2} \textit{[Cost c3]}

Recurrence relationship:

\begin{equation*}
  T(n)=\begin{cases}
    c, & \text{if $n=1$}.\\
    2T(n/2) + c, & \text{if n $>$ 1}.
  \end{cases}
\end{equation*}

Solve it:
\begin{align*}
T(n) &= 2T(n/2) + c\\
     &= 2^{lgn}T(1) + c\sum_{i=0}^{i=lgn-1}2^i \\
     &= nT(1) + c(n-1)
\end{align*}

\item Which of the algorithms Min1 (from Homework 2) or Min2 is faster?

\par Min1 algorithm:
\par \textbf{ALGORITHM} \textit{Min1}(A[0..n-1])
\par \textbf{if} n = 1 return A[0]  \quad [Constant time c1]
\par\textbf{else} \textit{temp} $\gets$ \textit{Min1}(A[0..n-2])  \quad  [c2 + T(n-1)]
\par \quad \quad \textbf{if} \textit{temp} $\leq$ A[n-1] \textbf{return} \textit{temp} \quad [Constant time c3]
\par \quad  \quad \textbf{else} \textbf{return} A[n-1] \quad [Constant time c4]
\par Recurrence relationship:
\begin{align*}
T(1) &= c1 \\
T(n) &= c + T(n-1)
\end{align*}

Solve it:
\begin{align*}
T(n) &= c + T(n-1) \\
     &= T(1) + (n-1)c \\
     &= \Theta(n)
\end{align*}
Both algorithms are $\Theta(n)$ so they are equal, however Min1 uses less assignments.


\end{enumerate}

\end{enumerate}
\end{document}
