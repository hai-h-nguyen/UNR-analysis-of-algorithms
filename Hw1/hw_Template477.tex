\documentclass[[11pts]{report}

\usepackage{qtree}
\usepackage{listings}
\usepackage{amsmath,mathtools}

\title{CS 677 Homework \\ Assignment 01}
\date{Semtember 4, 2018}
\author{Hai Nguyen}

\setlength{\topmargin}{-1cm}
\setlength{\oddsidemargin}{0in}
\setlength{\textwidth}{6.5in}
\setlength{\textheight}{8.3in}

\DeclareMathOperator{\Div}{div}
\newcommand{\vect}[1]{\mathbf{#1}}




%%Currently default settings for indentation and symbols.
%%Try these by uncommenting this block!!!
%%Redefine the first level symbols
%\renewcommand{\theenumi}{\fnsymbol{enumi}-}
%\renewcommand{\labelenumi}{\theenumi}
%
%%Redefine the second level symbols
%\renewcommand{\theenumii}{\alph{enumii})}
%\renewcommand{\labelenumii}{\theenumii}
%
%%Redefine the third level symbols
%\renewcommand{\theenumiii}{\roman{enumiii}.}
%\renewcommand{\labelenumiii}{\theenumiii}
%
%%Options for redefining levels


%\arabic
%\alph 
%\Alph
%\roman
%\Roman
%\fnsymbol
%This ^^^ is all you need to change!!

\begin{document}

\maketitle

\begin{enumerate}

\item Arrange the list of functions in ascending order of growth rate:

$f_1(n) = (n-1)!$
\\ $f_2(n) = 5lg(n + 100)^n$
\\ $f_3(n) = 2^{2n}$
\\ $f_4(n) = 0.001n^4 + 3n^3 + 1$
\\ $f_5(n) = ln^2n$ 
\\ $f_6(n) = \sqrt[3] {n}$
\\ $f_7(n) = 3^n$
\\ Ascending order:
$f_5(n) < f_6(n) < f_2(n) < f_4(n) < f_7(n) < f_3(n) < f_1(n)$ 

\item Using the informal definition for the $\Theta$ notation, select the correct notation for the following expression:

\begin{enumerate}

\item $2(lgn)^2 + 4n + 3n^2lgn = \Theta(n^2lgn)$

\item $(6n^3lgn + 4)(10+n) = \Theta(n^4lgn)$

\item $\frac{(n^2+lgn)(n+1)}{n+n^2} = \Theta(n)$

\item $ 2 + 4 + 8 + 16 + ... + 2^n = \Theta(2^n)$

\item $8^{lgn} = \Theta(n^3)$
\end{enumerate}

\item Using mathematical induction, show that the following relations are true for every $n \geq 1$

\begin{enumerate}

\item $\sum_{i=1}^{n}(-1)^{(i+1)}i^2 = \frac{(-1)^{n+1}n(n+1)}{2}$
\\ \\
\textit{Base case:} 
$n=1: (-1)^2\times1^2 = \frac{(-1)^{1+1}1(1+1)}{2}$ \textit{(True)} \\
\textit{Inductive case:} \\ \\
\textit{Assume:} $\sum_{i=1}^{n}(-1)^{(i+1)}i^2 = \frac{(-1)^{n+1}n(n+1)}{2}$ is True
\\ \\ \textit{Prove:} $\sum_{i=1}^{n+1}(-1)^{(i+1)}i^2 = \frac{(-1)^{n+2}(n+1)(n+2)}{2}$
 \\ \\ $\to \sum_{i=1}^{n}(-1)^{(i+1)}i^2 + (-1)^{n+2}(n+1)^2 =  \frac{(-1)^{n+2}(n+1)(n+2)}{2} $ 
\\ \\ $\to \frac{(-1)^{n+1}n(n+1)}{2} + (-1)^{n+2}(n+1)^2 =  \frac{(-1)^{n+2}(n+1)(n+2)}{2} $ 
\\ \\  $\to (-1)^{n+1}(n+1)(\frac{n}{2} + (-1)(n+1)) = \frac{(-1)^{n+2}(n+1)(n+2)}{2} $ 
\\ \\  $\to (-1)^{n+1}(n+1)(\frac{n-2n-2}{2}) = \frac{(-1)^{n+2}(n+1)(n+2)}{2} $
\\ \\ $\to  \frac{(-1)^{n+2}(n+1)(n+2)}{2} = \frac{(-1)^{n+2}(n+1)(n+2)}{2} $ \textit{(True)}

\item $\sum_{i=1}^{n}\frac{1}{(2i-1)(2i+1)} = \frac{n}{2n+1}$
\\ \\
\textit{Base case:} $n = 1: \frac{1}{1\times3} = \frac{1}{2\times1 + 1}$ \textit{(True)}

\textit{Inductive case:}
\\ \\
\textit{Assume:} $\sum_{i=1}^{n}\frac{1}{(2i-1)(2i+1)} = \frac{n}{2n+1}$ is True
\\ \\
\textit{Prove:} $\sum_{i=1}^{n+1}\frac{1}{(2i-1)(2i+1)} = \frac{n+1}{2n+3}$ \\

$\to \sum_{i=1}^{n}\frac{1}{(2i-1)(2i+1)} + \frac{1}{(2n+1)(2n+3)} = \frac{n+1}{2n+3}$ \\

$\to \frac{n}{2n+1} + \frac{1}{(2n+1)(2n+3)} = \frac{n+1}{2n+3}$ \\

$\to \frac{n(2n+3) + 1}{(2n+1)(2n+3)} = \frac{n+1}{2n+3}$ \\

$\to \frac{(n+1)(2n+1)}{(2n+1)(2n+3)} = \frac{n+1}{2n+3}$ \\

$\to \frac{n+1}{2n+3} = \frac{n+1}{2n+3}$ \textit{(True)} \\





\end{enumerate}

\item Using the formal definition of the asymptotic notations, prove the following statements:
\begin{enumerate}
\item $10n^2 + 1 \in O(n^3)$

Prove with $\exists c, n_0, \forall n \geq n_0$:
\begin{equation}
 10n^2 + 1 \leq cn^3
\label{eq1}
\end{equation}

We have:

\begin{equation}
10n^2 + 1 \leq 10n^3 + n^3 = 11n^3
\label{eq2}
\end{equation}

In order to satisfy Inequation \ref{eq1}, from Inequation \ref{eq2} we can choose $c = 11, n_0 = 1$.

\item $5n^2 + 10 \in \Omega(n)$

Prove with $\exists c, n_0, \forall n \geq n_0$:
\begin{equation}
5n^2 + 1 \geq cn
\label{eq3}
\end{equation}

We have:

\begin{equation}
5n^2 + 1 \geq 5n^2 \geq 5n
\label{eq4}
\end{equation}

In order to satisfy Inequation \ref{eq3}, from Inequation \ref{eq4} we can choose $c = 5, n_0 = 1$.

\end{enumerate}


\item Extra

\begin{itemize}
\item Find the order of growth for the following sum: $\sum_{i=1}^{n-1}(i+2)^2$
\\ We have:

\begin{equation*}
\sum_{i=1}^{n-1}(i+2)^2 = \sum_{i=1}^{n-1}{(i^2 + 4i + 4)} = \sum_{i=1}^{n-1}i^2 + 4\sum_{i=1}^{n-1}i + 4(n-1) = \frac{(n-1)n(2n-1)}{6} + 4\frac{(n-1)n}{2} + 4(n-1) = \Theta(n^3)
\end{equation*}

\item Use the formal definition of the asymptotic notation to prove that:

\begin{equation}
30n^2 + 100 \notin \Omega(n^3)
\label{eq0}
\end{equation}

Assume: 
\begin{equation*}
30n^2 + 100 \in \Omega(n^3)
\end{equation*}

It means $\exists c, n0$ so that with $\forall n \geq n0$:
\begin{equation}
30n^2 + 100 \geq cn^3
\label{eq5}
\end{equation}

We have:
\begin{equation}
30n^2 + 100 \leq 30n^2 + 100n^2 = 130n^2
\label{eq6}
\end{equation}

From Inequation \ref{eq5} and \ref{eq6}:

\begin{equation}
cn^3 \leq 130n^2 \to n \leq \frac{130}{c}
\label{eq7}
\end{equation}

Inequation \ref{eq7} cannot satisfy with $\forall n \geq n0$ so \ref{eq0} holds True.

\end{itemize}

\end{enumerate}


\end{document}
